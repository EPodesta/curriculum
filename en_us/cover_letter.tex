\documentclass[11pt, a4paper]{awesome-cv}

\geometry{left=1.4cm, top=.8cm, right=1.4cm, bottom=1.8cm, footskip=.5cm}
\colorlet{awesome}{awesome-red}
\setbool{acvSectionColorHighlight}{true}

\renewcommand{\acvHeaderSocialSep}{\quad\textbar\quad}

% \photo[circle,noedge,left]{./profile}
\name{Emmanuel}{Podestá Junior}
\position{Software Engineer}

\email{epodestaj@gmail.com}
\github{EPodesta}
\linkedin{epodestaj}

\quote{``All we have to decide is what to do with the time that is given to us."}


%-------------------------------------------------------------------------------
%	LETTER INFORMATION
%	All of the below lines must be filled out
%-------------------------------------------------------------------------------
% The company being applied to
\recipient
  {Company Recruitment Team}
  {Canonical}
% The date on the letter, default is the date of compilation
\letterdate{\today}
% The title of the letter
\lettertitle{Job Application for Software Engineer, Integration QA, Python}
% How the letter is opened
\letteropening{Dear all,}
% How the letter is closed
\letterclosing{Sincerely,}
% Any enclosures with the letter
\letterenclosure[Attached]{Curriculum Vitae}


%-------------------------------------------------------------------------------
\begin{document}

% Print the header with above personal information
% Give optional argument to change alignment(C: center, L: left, R: right)
\makecvheader[R]

% Print the footer with 3 arguments(<left>, <center>, <right>)
% Leave any of these blank if they are not needed
\makecvfooter
  {\today}
  {Emmanuel Podestá Junior~~~·~~~Cover Letter}
  {}

% Print the title with above letter information
\makelettertitle

%-------------------------------------------------------------------------------
%	LETTER CONTENT
%-------------------------------------------------------------------------------
\begin{cvletter}

\lettersection{About Me}

I have 7+ years experience on parallel and distributed systems focusing on low
level development. With my experience, I learned to tackle several tasks to
improve problem-solving skills, and increase my knowledge about tools and
technologies. In each of those years, I developed and increased my love for
challenging tasks and the knowledge that comes with every task solved.

I started my love for learning Computer Science in 2015, when I began my
Research and Development project in Universidade Federal de Santa Catarina
(UFSC). There I began to improve my skills working with several colleagues and
contributing with them in a project. As time goes by, I began to present works
based on the project in many conferences and, as the feedback arrives, the
project began to improve even more. At the end of my academic undergraduate
education, I had worked for many years with C++; had a solid experience with the
Linux environment and High Performance Computing; and improved my
problem-solving and communication skills.  After this experience I decided to
improve my academic education with a master's degree. I worked with C and Python
to analyze and improve a Distributed Operating System. Every step to improve in
this project was followed by communicating with colleagues and selecting the
best approach for many problems.

These two projects were open source and have a great length of contributors from
many places. Furthermore, during my master's education, the project was highly
associated with several other projects in the Operating System subject and had
several international contributors. I had the opportunity to work abroad for a
University and collaborated with authorities from the project.

There were many other topics approached in an extracurricular manner, such as
MySQL, High Performance Computing, Parallel environments, Java, and many other
technologies that are comprehensively listed in my resume.

After some years I finished my education and went in search of new horizons. I
began to study WordPress, PHP, Javascript, HTML and CSS to learn more about the
world of web development.

\lettersection{Why Canonical?}
I always believed in improving peoples lives, even with the tiniest of  actions.
When I read about Canonical, I recognized how many lives it can reach and how
much it can improve people's day-to-day lives.

Since my first days in my academic education I fell in love with open source and
the collaborative power it has.  Many different people working on the same
project striving towards a goal is incredible. I love the feeling of explaining
a problem and finding the solution of an issue, the brainstorm of ideas to solve
a specific problem or, even, getting a Pull Request accepted after some
discussion about better ways to improve it. All of this combined with an active
community, a software that can always be improved freely step-by-step to achieve
a final goal and a inclusive environment is really alluring.

Finally, I love the Linux environment. I may spend hours configuring my own
Operating System to fit my individuality. To work in the company responsible for
Ubuntu and several other technologies is really incredible.

\lettersection{Why Me?}
During my +7 years of experience in developing solutions to parallel and
distributed systems, I improved a parallel framework power efficiency;
introduced a solution that provides means to evaluate features inside a open
source Distributed Operating System even without the support for it; and worked
with several people from many countries. In these experiences, I developed my
problem-solving skills, introduced several tests for each project, learned how
to communicate concisely and work in a cooperative environment (even remotely).

Furthermore, I want to work with open source projects and contribute to them. In these
projects, I want to work with talented people, contribute for worldwide
technologies and work with a concise communication to provide a better workflow
to reach milestones in a project. With each goal being accomplished, I want to
see the project being improved and witness its impact on people.

Finally, I can bring the knowledge from testing High Performance Computing
projects, open source contribution experience, concise communication and a
highly motivated behavior to surpass every challenge.

\end{cvletter}


%-------------------------------------------------------------------------------
% Print the signature and enclosures with above letter information
\makeletterclosing

\end{document}
